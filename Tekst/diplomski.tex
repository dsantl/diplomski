\documentclass[times, utf8, diplomski, numeric]{fer}
\usepackage{booktabs}

\begin{document}

% TODO: Navedite broj rada.
\thesisnumber{656}

% TODO: Navedite naslov rada.
\title{Bojanje grafova prilagodljivim metaheurističkim postupcima}

% TODO: Navedite vaše ime i prezime.
\author{Dino Šantl}

\maketitle

% Ispis stranice s napomenom o umetanju izvornika rada. Uklonite naredbu \izvornik ako želite izbaciti tu stranicu.
\izvornik

% Dodavanje zahvale ili prazne stranice. Ako ne želite dodati zahvalu, naredbu ostavite radi prazne stranice.
\zahvala{}

\tableofcontents

\chapter{Uvod}
Uvod rada. Nakon uvoda dolaze poglavlja u kojima se obrađuje tema.

\section{Sažetci korištenih znanstvenih članaka}

\chapter{Formalni opis problema}

U ovom poglavlju opisuje se problem bojanja grafova i njegova primjena u telekomunikacijskom problemu koji se rješava. Najprije se predstavlja problem u telekomunikacijama uz pretpostavke i ograničenja. Zatim se isti problem opisuje pomoću matematičkog modela tj. problema bojanja grafova. Prvo se objašnjava općeniti problem bojanja grafova, a zatim se matematički model prilagodi konkretnom problemu.

\section{Tehnički opis problema}

Za shvaćanje problema dovoljno je pretpostaviti da postoji korisnikova oprema (što je najčešće mobilni uređaj) i bazna stanica. Bazne stanice geografski su statične, za razliku od mobilnih uređaja.
Radio pristupna mreža \engl{Radio Access Network} (RAN) dio je telekomunikacijskog sustava i nalazi se između opreme korisnika i jezgre mreže. Ne ulazeću u detalje, za ovaj problem dovoljno je reći da je to bazna stanica na koju se spaja korisnikova oprema (tj. mobilni uređaj).

Za što kvalitetnji rad mreže potrebno je optimirati parametre \emph{RAN}-a. Jedan od parametra je \emph{scrambling} kod. 


\section{Matematički opis problema}

\chapter{Infrastruktura}

\chapter{Algoritmi}

\chapter{Testiranje}

\chapter{Rezultati}

\chapter{Zaključak}
Zaključak.

\bibliography{literatura}
\bibliographystyle{fer}

\begin{sazetak}
Sažetak na hrvatskom jeziku.

\kljucnerijeci{Ključne riječi, odvojene zarezima.}
\end{sazetak}

% TODO: Navedite naslov na engleskom jeziku.
\engtitle{Title}
\begin{abstract}
Abstract.

\keywords{Keywords.}
\end{abstract}

\end{document}
