\documentclass[times, utf8, diplomski, numeric]{fer}
\usepackage{booktabs, url, hyperref}
\usepackage{amsthm}

\hypersetup{
   colorlinks,
   citecolor=black,
   filecolor=black,
   linkcolor=black,
   urlcolor=black
}


\newtheorem{definition}{Definicija}
\newtheorem{theorem}{Teorem}


\begin{document}

% TODO: Navedite broj rada.
\thesisnumber{656}

% TODO: Navedite naslov rada.
\title{Bojanje grafova prilagodljivim metaheurističkim postupcima}

% TODO: Navedite vaše ime i prezime.
\author{Dino Šantl}

\maketitle

% Ispis stranice s napomenom o umetanju izvornika rada. Uklonite naredbu \izvornik ako želite izbaciti tu stranicu.
\izvornik

% Dodavanje zahvale ili prazne stranice. Ako ne želite dodati zahvalu, naredbu ostavite radi prazne stranice.
\zahvala{}

\tableofcontents

\chapter{Uvod}
Uvod rada. Nakon uvoda dolaze poglavlja u kojima se obrađuje tema.

\section{Sažetci korištenih znanstvenih članaka}

\chapter{Formalni opis problema}

U ovom poglavlju opisuje se problem bojanja grafova i njegova primjena u telekomunikacijskom problemu koji se rješava. Najprije se predstavlja problem u telekomunikacijama uz pretpostavke i ograničenja. Zatim se isti problem opisuje pomoću matematičkog modela tj. problema bojanja grafova. Prvo se objašnjava općeniti problem bojanja grafova, a zatim se matematički model prilagodi konkretnom problemu.

\section{Tehnički opis problema}

Za shvaćanje problema dovoljno je pretpostaviti da postoji korisnikova oprema (što je najčešće mobilni uređaj) i bazna stanica. Bazne stanice geografski su statične, za razliku od mobilnih uređaja.
Ono što se u ovom radu naziva baznom stanizom je radio pristupna mreža \engl{Radio Access Network} (\emph{RAN}). \emph{RAN} dio je telekomunikacijskog sustava i nalazi se između opreme korisnika i jezgre mreže. To je sloj u mreži koji je zaslužan za prenošenje komunikacije između mobilnog uređaja do antene i od antene do drugog mobilnog uređaja. 

Za što kvalitetnji rad mreže potrebno je optimirati parametre \emph{RAN}-a. Jedan od parametra je \emph{scrambling} k\^{o}d. \emph{Scrambling} k\^{o}d međuostalim služi kako bi mobilni uređaj mogao razlikovati bazne stanice. Zbog toga je potrebno svakoj baznoj stanici pridružiti različit k\^{o}d. Problem je u tome što je dostupno samo 512 različitih kodova. To znači da neke bazne stanice moraju imati isti k\^{o}d (ako se u sustavu nalazi više od 512 baznih stanica). Ako neke dvije bazne stanice imaju isti k\^{o}d i mobilni uređaj se nalazi u dosegu signala od obje stanice, tada ne zna s kojom baznom stanicom komunicira te pada kvaliteta usluge.

Potrebno je dodjeliti \emph{scrambling} k\^{o}dove tako da dvije bazne stanice koje mogu biti istovremeno vezane za jedan mobilni uređaj nemaju isti k\^{o}d ili ako je to nemoguće smanjiti, dodjeliti k\^{o}dove tako da je negativan utjecaj na kvalitetu usluge što manji. Postoje još neka tehnička ograničenja koja je potrebno uvažiti, a biti će navedena u nastavku.

Iako postoji 512 različitih k\^{o}dova, svaka bazna stanica ima ograničen skup k\^{o}dova koje može koristiti. Dakle, za svaku baznu stanicu zadan je skup k\^{o}dova. Za neke bazne stanice postoji svojstvo nepromjenjivost, što znači da k\^{o}d koji je trenutno zadan za tu baznu stanicu mora ostati takav. Takve se vrste baznih stanica nazivaju \textbf{nepromjenjivima}. Početno stanje mreže definirano je k\^{o}dovima koji su pridjeljeni nekoj baznoj stanici. Osim toga za svaku baznu stanicu poznata je njezina vrsta. Vrsta može biti označena sa slovima: \emph{A, B ili C}. Bazne stanice različitih vrsta ne utjeću jedna na drugu.

\subsection{Ulazni i izlazni podaci}

Kao ulaz algoritam dobije inicjalno stanje mreže. Odnosno za svaku baznu stanicu poznat je trenutni k\^{o}d koji koristi i sva pravila koja moraju biti zadana. Izlaz algoritma je popis oznaka baznih stanica i za svaku od oznaka pripadajući k\^{o}d.

\subsubsection{Ulazni podaci}

\begin{enumerate}
	\item Popis i definicija domene boja (skupova boja)
	\item Popis čvorova (baznih stanica)
		\begin{enumerate}
			\item Oznaka čvora
			\item Vrsta čvora (grupa) - A, B ili C
			\item Oznaka domene za boju čvora (koji skup boja koristi)
			\item Početna boja za čvor
			\item Oznaka da li je čvor nepromjenjiv
		\end{enumerate}
\end{enumerate} 

\subsubsection{Izlazni podaci}

Kao izlaz algoritma koristi se par brojeva $(i, c)$ gdje je $i$ oznaka za čvor, a $c$ je boja čvora.

\section{Matematički opis problema}

Potrebno je modelirati problem u kojem postoji \emph{bazna stanica} i veze između istih, koje predstavljaju mjeru u kojoj jedna bazna stanica utjeće na drugu. Jedan od mogućih modela je graf. Čvorovi će predstavljati bazne stanice, a jakost između dvije bazne stanice biti će modelirano pomoću težine brida. U nastavku bazna stanica nazivat će se čvorom. Nad ovako postavljenim grafom problem je dodjeliti k\^{o}dove tako da ako je moguće nema bridova koji na svojim krajevima imaju čvor s istim k\^{o}dom. Problem je poznat pod nazivom \emph{bojanje grafa}. Kako je uobičajeno pričati o bojama čvora (a ne o k\^{o}dovima) od sad pa nadalje \emph{scrambling} k\^{o}d nazivat će se bojom čvora. U nastavku će prvo biti opisan klasičan problem bojanja grafova, gdje se promatraju beztežinski grafovi (oni čiji bridovi nemaju težine). Nakon toga problem se poopćuje na težinske grafove, čime se modelira predhodno opisan problem.

\subsection{Definicije za klasičan problem bojanja grafova}

Najprije se definiraju matematički pojmovi. Nakon toga formalno je opisan problem bojanja grafova. Uz to se nadovezuje teorija izračunljivosti.

\subsubsection{Pojam grafa}

\begin{definition}
Jednostavni graf \textbf{G} sastoji se od nepraznog konačnog skupa \textbf{V(G)}, čije elemente nazivamo čvorovi grafa \textbf{G} i konačnog skupa \textbf{E(G)} različitih dvočlanih podskupova \textbf{V(G)} koje zovemo bridovi.
\end{definition}

\begin{definition}
Skup vrhova koji su susjedni vrhu \textbf{v} zovemo susjedstvo vrha \textbf{v} i označavamo s oznakom \textbf{H(v)}.
\end{definition}

\begin{definition}
Stupanj vrha \textbf{v} grafa \textbf{G} jednak je broju bridova koji su incidentni s \textbf{v}. Označavamo ga s \textbf{d(v)}.
\end{definition}

\begin{definition}
Neka je \textbf{G} jednostavan graf i $\mathbf{\omega}$ funkcija $\mathbf{\omega : E \rightarrow \mathbb{R}}$. Par \textbf{(G, $\omega$)} naziva se težinski graf. Pri čemu funkcija $\omega$ svakom bridu iz \textbf{G} dodjeljuje jedan element iz skupa realnih brojeva. Neka je \textbf{e} neki brid grafa \textbf{G}, njegova težina označena je s $\omega(\mathbf{\mathit{e}})$.	 
\end{definition}


\subsubsection{Definicija bojanja grafova}

\begin{definition}
Definiramo funkciju $\phi : V(G) \rightarrow \mathbb{N}$, koja svakom čvoru u grafu pridružuje jedan prirodan broj koji $\mathbf{\phi(v)}$, gdje je $\mathbf{v}$ čvor u grafu $\mathbf{G}$. Broj $\mathbf{\phi(v)}$ nazivamo boja čvora, a funkciju $\phi$ nazivamo \textbf{bojanje grafa}. 
\end{definition}

\begin{definition}
Bojanje grafa s najviše \textbf{k} boja nazivamo \textbf{k-bojanje} grafa. 
\end{definition}

\begin{definition}
Ako se graf može obojati s najviše \textbf{k} boja tada takvo bojanje nazivamo \textbf{legalno k-bojanje} grafa. 
\end{definition}

\begin{definition}
Graf je \textbf{k-obojiv} akko postoji legalno k-bojanje grafa.
\end{definition}


\begin{definition}
Ako je graf G \textbf{k}-obojiv, ali nije \textbf{(k-1)}-obojiv tada kažemo da je \textbf{k} kromatski broj grafa G, gdje se koristi oznaka $\chi(G)=k$  
\end{definition}


\begin{definition}
Podskup skupa V(G) nazivamo \textbf{nezavisni skup} ako u njemu ne postoje dva čvora koja su susjedna.  
\end{definition}

\begin{definition}
Brid koji spaja dva čvora iste boje nazivamo \textbf{konfliktni brid}.
\end{definition}

\begin{definition}
Dva čvora koja spaja konfliktni brid nazivamo \textbf{konfliktnim čvorovima}.
\end{definition}

\begin{definition}
Particiju skupa V(G) na k disjunktnih nepraznih podskupova $V_1,..., V_k$
 tako da vrijedi $V(G) = \bigcup_{j=1}^{k}V_j$ zovemo \textbf{k-dioba} grafa G. Ako su podskupovi  $V_1 ,..., V_k$  ujedno i nezavisni skupovi onda se to naziva \textbf{legalna k-dioba} grafa G.
\end{definition}

\begin{theorem}
\label{thm:dioba}
Graf \textbf{G} je \textbf{k-obojiv} akko postoji \textbf{legalna k-dioba} grafa \textbf{G}.
\end{theorem}

\begin{proof}
Pretpostavimo da je graf \emph{G} \textbf{k-obojiv}. Definiramo skupove $S_i$ tako da čvor grafa $\mathbf{v}$ pripada skupu $S_i$ ako je obojan bojom $\mathbf{i}$. Takvi skupovi su neprazni i disjunktni, te unija skupova $S_i$ čini skup svih vrhova grafa \emph{G}. Kako je G \textbf{k-obojiv} tada ne postoji boja zbog koje bi neki brid bio konfliktan pa sljedi da je podjela na skupove $S_i$ \textbf{legalna k-dioba} grafa G jer su skupovi $S_i$ nezavisni. 

Drugi smjer dokazuje se tako što se pretpostavi da postoji \textbf{legalna k-dioba} grafa \textbf{G} na skupove $S_i$. Kako je svaki skup indeksiran s indeksom \textbf{i}, tada svakom čvoru koji se nalazi u skupu $S_i$ dodjelimo boju \textbf{i}. Kako su skupovi $S_i$ prema pretpostavci disjunktni, neprazni i nezavisni tada ne postoje dva čvora koja bi imala istu boju. Q.E.D
\end{proof}

\subsubsection{Definicije računske teorije složenosti}

\begin{definition}
\textbf{Problem odluke} je problem koji uvijek ima odgovor \textbf{da} ili \textbf{ne}. Primjerice, problem da li je graf točno \textbf{k-obojiv} je problem odluke. Problem traženja kromatskog broja grafa nije problem odluke, jer tražimo točan broj \textbf{k}.
\end{definition}

\begin{definition}
Problem odluke za koje postoje algoritmi koji daju odgovor, a čije vrijeme izvršavanje ovisi polinomno o veličini ulaznih podataka spadaju u \textbf{klasu P}.
\end{definition}

\begin{definition}
\textbf{Problem odluke} spada u klasu \textbf{NP problema} ako se točnost njegovog rješenja može ispitati u polinomnom vremenu. 
\end{definition}

\begin{definition}
\textbf{Problem odluke} spada u klasu \textbf{NP-potpunih problema} ako spada u \textbf{klasu NP} problema i koji ima svojstvo da se svaki drugi problem iz \textbf{klase NP} može polinomno reducirati na njega. 
\end{definition}

\begin{definition}
Kažemo da je problem \textbf{NP-težak} akko postoji \textbf{NP-potpun problem} koji se može polinomno reducirati na njega.
\end{definition}

Problem odluke za koji postoji algoritam čije trajanje ovisi polinomno o veličini ulaznih podataka tada je to \textbf{P problem}. Ako za neki problem odluke ne možemo pronaći algoritam čije vrijeme izvođenja ovisi polinomno o veličini ulaznih podataka, ali točnost rješenja možemo provjeriti u polinomnom vremenu tada je to \textbf{NP problem}. \textbf{NP-potpun problem} je problem odluke čiji algoritam možemo iskoristiti da bi rješili sve \textbf{NP probleme} tako da koristimo polinoman broj poziva tog algoritma. Općeniti problemi (koji ne moraju biti problemi odluke) nazivaju se \textbf{NP-teški problemi} ako postoji problem u klasi \textbf{NP-potpunih problema} koji se može rješiti pomoću polinomnog broja poziva algoritma za promatrani \textbf{NP-teški} problem. Odnost \textbf{P} i \textbf{NP} klasa još je uvijek otvoren problem ($P=NP$ ili $P\subset NP$). Kada bi se moglo dokazati da je neki \textbf{NP-potpun} problem moguće rješiti u polinomnom vremenu, tada bi se svi \textbf{NP problemi} mogli rješiti u polinomnom vremenu, pa bi klase \textbf{P} i klase \textbf{NP} bile jednake. Ako se pak pokaže da za neki \textbf{NP} problem ne postoji algoritam čije izvršavanje ovisi polinomno o ulaznim podacima, tada bi klasa \textbf{P} bila pravi podskup od klase \textbf{NP}.  

\subsubsection{\emph{No free lunch} teorem za optimizacijske algoritme}

\begin{definition}
\textbf{Optimizacija} je grana matematike koja proučava pronalaženje ekstrema funkcija.
\end{definition}

\begin{definition}
\textbf{Kombinatorna optimizacija} je grana optimizacije, gdje je domena funkcije skup s konačnim brojem elemenata.
\end{definition}

\begin{definition}
Konkretna funkcija koja se proučava naziva se \textbf{funkcija cilja}.
\end{definition}

\begin{definition}
Funkcije cilja koje se promatraju imaju diskretnu i konačnu domenu i kodomenu. Iako se kao elementi koodomene mogu pojaviti realni brojevi, zbog toga što su računala diskretni strojevi s konačno mnogo memorije, to je samo konačni podskup realnih brojeva.
\end{definition}

\begin{theorem}
Funkcija cilja definira se kao: $f : \mathbb{X} \rightarrow \mathbb{Y}$, gdje su skupovi $\mathbb{X}$ i $\mathbb{Y}$ diskretni i konačni. Tada je broj svih mogućih funkcija jednak $|\mathbb{F}| = |\mathbb{Y}|^{|\mathbb{X}|}$. 
\end{theorem}

\begin{proof}
Za svaki element domene možemo odabrati točno $|\mathbb{Y}|$ elemenata koodomene. 
\begin{equation}
|\mathbb{F}| = \prod_{j=1}^{|\mathbb{X}|} |\mathbb{Y}| =  |\mathbb{Y}|^{|\mathbb{X}|}
\end{equation}
\end{proof}

\begin{definition}
Vremenski niz $d_m$ je niz parova domene i kodomene.
\begin{equation}
d_m = \left \{ (d_m^x(1), d_m^y(1)), (d_m^x(2), d_m^y(2)), ..., (d_m^x(m), d_m^y(m))\right \}
\end{equation} 
Iz nekog niza $d_m$ može se izvući samo niz elemenata domene i to se označava s $\mathbf{d_m^x}$ ili samo niz elemenata kodomene što se označava s $\mathbf{d_m^y}$.
\end{definition}

\begin{definition}
Prostor svih vremenskih nizova veličine $m$: $\mathbb{D}_m = (\mathbb{X} \times \mathbb{Y})^m$, a prostor svih vremenskih nizova maksimalne veličine $m$ je: $\mathbb{D} = \bigcup_{m\ge0}\mathbb{D}_m$.
\end{definition}

\begin{definition}
Optimizacijski algoritam $a$ definira se kao: $a : d \in \mathbb{D} \rightarrow \left \{x | x \notin d_m^x \right \}$
\end{definition}

Optimizacijski algoritam preslikava neki vremenski niz (podatke iz predhodnih koraka izvođenja) u novu vrijednost domene. S time da ta vrijednost domene ne smije biti već viđena. Ovo ograničenje koristi se u dokazu teorema, ali u praksi se često mogu javiti slučajevi gdje se više puta računa vrijednost funkcije za isti element domene.

\begin{definition}
\textbf{Heuristikom} se naziva optimizacijski algoritam koji ne mora kao rezultat dati globalni ekstram \textbf{funkcije cilja}, ali daje dovoljno dobre rezultate u svrhu bržeg kraćeg vremenskog izvođenja. 
\end{definition}

\begin{definition}
\textbf{Metaheuristikom} naziva se familija optimizacijskih algoritama, čije konkretno izvođenje ovisi o njegovim parametrima.
\end{definition}

\begin{definition}
Skup svih dozvoljenih rješenja u optimizaciji nazivamo \textbf{prostor pretrage}.
\end{definition}

\begin{theorem}
\label{thm:nofreelunch}
\textbf{"No free lunch" teorem za optimizaciju} - Neka su $a_1$ i $a_2$ dva različita heuristička algoritma koji traže ekstrem funkcije. Funkcije su predstavljene crnom kutijom. Tada vrijedi ova jednakost:
\begin{equation}
	\sum_{f}P(d_m^y|f,m,a_1) = \sum_{f}P(d_m^y|f,m,a_2)
\end{equation}
\end{theorem}

Dokaz teorema može se pronaću u [literatura].

Kako se ne zna ništa više o funkcijama (npr. simbolički zapis) već samo ulazni i izlazni parovi ne može se koristiti nikakvo unutarnje znanje koje bi nekom algoritmu dalo prednost. Ako je poznat broj koraka $m$ tada za bilo koji algoritam, sume vjerojatnosti su međusobno jednake, pri čemu se gleda vjerojatnost da se pojavio neki niz izlaznih vrijednosti funkcije $d_m^y$.
Iz niza izlaznih vrijednost $d_m^y$ može se lako izvući minimalna ili maksimalna vrijednost koja je tada konačan izlaz algoritma. Dakle ne postoji algoritam koji bi bio dominantniji za sve funkcije cilja $f$. Zbog toga je potrebno svaki optimizacijski problem promatrati veoma usko i koristiti dodatne informacije o samoj funkciji $f$.

Predhodne definicije i teoremi služe kako bi se uskladili nazivi za matematičke pojmove koji će se koristiti u radu. Sve definicije uzete su iz [literatura], gdje se međuostalim mogu naći i druge definicije i teoremi vezani za teoriju grafova, računsku teoriju složenosti i "\emph{No free lunch}" teorem.

\subsection{Bojanje težinskih grafova}

Za navedeni telekomunikacijski problem potrebno je odabrati prikladan matematički model. Već je spomenuto da se bazne stanice modeliraju čvorovima u grafu. Postavlja se pitanje što predstavlja brid u tome grafu. Očigledno je da jedna bazna stanice utjeće na drugu baznu stanicu, tj. mobilni uređaj istovremeno vidi više baznih stanica. To znači da bi brid mogao predstavljati relaciju da li neka bazna stanica vidi drugu baznu stanicu. Ali problem se može razviti i korak dalje. Nije svejedno na kojoj su udaljenosti bazne stanice, te koja je njihova snaga. To znači da dvije bliže (ili jače) bazne stanice više utjeću jedna na drugo nego što to čine dvije udaljenije. Idaja je svakom bridu dodijeliti težinu koja predstavlja neku mjeru koliko jedna bazna stanica utjeće na drugu. To znači da brid između dva čvora daje mjeru međusobnog utjecaja čvorova. Ako je utjecaj premali (tehnički nevidljiv), tada brid između ta dva čvora ne postoji.

Na tako zadanom težinskom grafu potrebno je napraviti bojanje grafa tako da se poštuju tehnički uvjeti. Sad se pojavio problem jer bojanje težinskog grafa nije dobro definiran problem. Zato se klasični problem bojanja grafova poopćuje:

\begin{definition}
Bojanje težinskog grafa je optimizacijski problem u kojem se minimizira funkcija cilja:
\begin{equation}
f(\phi) = \sum_{i=1}^{E(G)} \omega(e_i) \cdot R(e_i)
\end{equation}
, gdje je $\omega(e_i)$ težina brida, a $R$ funkcija koja ima vrijednost $1$ ako je $e_i$ konfliktan brid ili $0$ inače. Funkcija $f$ ovisi o bojanju grafa, tj. funkcija cilja jednaka je sumi težina konfliktnih bridova.
\end{definition}

Zbog tehničkih ograničenja potrebno je uvesti još malo matematičkih detalja koji pokrivaju te slučajeve.

\begin{definition}
Funkcija $\phi$ definira se kao funkcija koja svakom ne nepromjenjivom čvoru $v$ grafa \textbf{G} pridružuje prirodan broj iz skupa dopuštenih boja za taj čvor.
\end{definition}

\begin{definition}
Ako su dva čvora \textbf{nepromjenjiva} spojena bridom i imaju iste boje, tada smatramo da taj brid nije \textbf{konfliktan}.
\end{definition}

Razlog ovakvoj definiciji je to što bridovi koji spajaju dva nepromjenjiva čvora različitih boja ne mogu se nikada poboljšati, pa shodno tome neovisni su o funkciji bojanja grafa $\phi$.

\begin{definition}
Ako su dva čvora koja spadaju u suprotne vrste spojena bridom, tada se takav brid ne smatramo \textbf{konfliktnim}.
\end{definition}

Tehnički takva dva čvora ne utjeću jedan na drugog pa tada nema potrebe to ugrađivati u funkciju cilja.

U slučaju da je čvor obojan bojom koja nije u njegovoj domeni tada je potrebno dodati u funkciju cilja kaznu za takav slučaj. Iako po definiciji funckije bojanja grafa to nije moguće, zbog inicjalnog stanja boja koje algoritam primi kao ulaz može se dogoditi da boja čvora nije u njegovoj domeni. Stoga se definira proširena funkcija cilja kao:

\begin{definition}
\begin{equation}
\label{equ:funkcija_cilja}
f(\phi) = \sum_{i=1}^{E(G)} \omega(e_i) \cdot R(e_i) + \sum_{i=1}^{V(G)} C(v_i)
\end{equation}
Prva suma jednaka je već definiranoj funkciji cilja. Drugi član je suma funkcije $C$ po čvorovima, gdje funkcija $C$ ima vrijednst $\varepsilon$ ako je čvor $v_i$ krivo obojan ili $0$ ako je obojan bojom iz svoje domene.
\end{definition}

Za epsilon se odabire neki veliki pozitivan broj. U konkretnoj implementaciji koja se koristi u radu $\varepsilon=10000000$.

\begin{definition}
Smatra se da je bojanje \textbf{valjano} ako postotak promjene boja inicjalnog stanja ne prelazi prag od $\alpha$ posto.
\end{definition}

U konkretnom problemu $\alpha=66\%$.

Sažetak problema kojeg modeliramo pomoću težinskog grafa glasi:
\begin{itemize}
	\item Svaki čvor može poprimiti jednu boju iz skupa dopuštenih boja za taj čvor
	\item Svaki čvor spada u jednu grupu tj. vrstu (A, B ili C)
	\item Neki čvorovi su definirani kao nepromjenjivi, njima se boja nikad ne smije mjenjati
	\item Dodatan uvijet je da postotak promjenjenih čvorova naprema inicijalnom stanju ne smije biti veća od 66\%
	\item Optimizacija se provodi nad funkcijom cilja definiranom formulom (\ref{equ:funkcija_cilja}).
\end{itemize}

\section{Analiza problema bojanja težinskih grafova}

U ovom odjelku biti će analiziran zadani matematički problem optimizacije. U svakom koraku analize uspoređuju se klasičan problem bojanja grafova i bojanje težinskih grafova. Kreće se od analize prostora pretrage. Zatim se dokazuje u kojoj se klasi problemi nalaze. Na kraju se pokazuje da "\emph{No free lunch}" teorem vrijedi i za bojanje grafova i kakve posljedice donosi.

\subsection{Prostor pretrage stanja}

Kod klasičnog problema bojanja grafova analiziraju se dva problema. Prvi je odrediti da li se graf može pobojati s najviše $k$ boja, a drugi problem je odrediti najmanji takav $k$. Pretpostavimo za početak da provjeravano da li je graf \textbf{k-obojiv}. Pitanje je na koliko načina se graf može pobojati ako koristimo najviše \textbf{k} boja. Svaki čvor može poprimiti \textbf{k} boja. To znači da je ukupan broj bojanja $N$ jednak: 

\begin{equation}
N = k^{V(G)} 
\end{equation}

Valja primjetiti da u ovom brojanju veličine prostora stanja postoji više jednako vrijednih bojanja. U bojanju nije bitna točna boja, već je bitno da različiti skupovi čvorova imaju različite boje. Točno pitanje je na koliko načina možemo čvorove podijeliti u $k$ skupova (teorem \ref{thm:dioba}).  

\begin{equation}
N = \sum_{i=1}^{k} S(V(G), i)
\end{equation}

Oznaka $S(m,n)$ je za \emph{Stirlingov} broj druge vrste, gdje je $m$ broj različitih elemenata (čvorovi) koje smještamo u $n$ istovrsnih skupova boja, tako da svaka boja ima barem jedan čvor. Kako je dopušteno da kutije mogu biti prazne, tada se problem pretvori u disjunktne probleme gdje se koristi samo $i$ kutija. 

Ako se zahtjeva da svaka od $k$ boja mora biti iskorištena, ekvivalentno da ni jedan od $k$ skupova ne smije biti prazan, tada je ukupan broj stanja nešto manji:

\begin{equation}
N = S(V(G), k) = \frac{1}{k!}\sum_{i=0}^{k} (-1)^i \binom{k}{i}(k-i)^{V(G)}  
\end{equation}

Ako se čvrsto zahtjeva da svaka boja mora biti barem na jednome čvoru, tada je to \emph{Stirlingov} broj druge vrste, gdje je i navedena formula za izračun $S(m,n)$.

Neka se promatra problem traženja kromatskog broja grafa $\chi(G)$, tj. minimalni broj boja $k$, a da se graf može legalno obojiti. Maksimalni broj za koji je to potrebno provjeriti je broj čvorova u grafu, zato jer u tom slučaju svaki čvor povezan je sa svakim drugim i svaki čvor mora imati svoju boju. Kromatski broj grafa je dakle ograničen odozgo s brojem čvorova. Za svaki broj $k$ do $V(G)$ treba provjeriti da li je graf \textbf{k-obojiv}. Ukupan broj stanja je:

\begin{equation}
N = B(V(G)) = \sum_{i=0}^{V(G)} S(V(G), i)
\end{equation}

Broj $B(n)$ naziva se \emph{Bellov} broj i on predstavlja broj načina na koje se skup od $n$ članova može podijeliti u neprazne podskupove. Za bolji uvid koliko brzo \emph{Bellov} niz raste dane su nejednakosti (za $n\ge 8$): 

\begin{equation}
2^n \le B(n) \le n! 
\end{equation}

\emph{Bellov} broj brže raste od eksponencijalne funkcije, ali sporije od faktorijela.

\begin{proof}
Prvo se pokazuje nejednakost: $2^n\le B(n), n \ge 5$.
Dokaz se provodi indukcijom. Koriste se razvoj binoma: 
\[ 2^n=(1+1)^n = \sum_{k=0}^{n}\binom{n}{k} \] 
i svojstvo Bellovog broja: 
\[ B(n+1)=\sum_{k=0}^{n}\binom{n}{k}B(k) \]
	\begin{align}
		2^n \le B(n) \\
		\textup{Baza indukcije: } n = 5, 2^5 \le B(5) \rightarrow 32 \le 52 \\
		\textup{Pretpostavka: } 2^n \le B(n) \\
		2^{(n+1)} \le B(n+1) \\
		2\cdot 2^n \le B(n+1) \\
		\sum_{k=0}^{n} 2\cdot \binom{n}{k} \le \sum_{k=0}^{n}\binom{n}{k}B(k)\\
		\sum_{k=0}^{3} 2\cdot \binom{n}{k} + \sum_{k=4}^{n} 2\cdot \binom{n}{k} \le \sum_{k=0}^{3}\binom{n}{k}B(k) + \sum_{k=4}^{n}\binom{n}{k}B(k) \\
		\textup{Prvo se dokazuje: } \sum_{k=0}^{3} 2\cdot \binom{n}{k} \le \sum_{k=0}^{3}\binom{n}{k}B(k) \\
		2\binom{n}{0} + 2\binom{n}{1} + 2\binom{n}{2} + 2\binom{n}{3} \le \\ \binom{n}{0}B(0) + \binom{n}{1}B(1) + \binom{n}{2}B(2) + \binom{n}{3}B(3) \\
		2+2n+\frac{n(n-1)}{2}2+\frac{1}{3}n(n-1)(n-2) \le \\1 + n + \frac{n(n-1)}{2}2+\frac{n(n-1)(n-2)}{6}5 \\
		1+n \le n(n-1)(n-2)(\frac{5}{6}-\frac{1}{3}) \\
		1+n \le \frac{n(n-1)(n-2)}{3} \\
		3+3n\le n^3 - 3n^2 +3n \\
		3n^2+3 \le n^3 \\
		3 \le n^3 - 3n^2 \\
		3 \le n^2 (n - 3), \textup{istina za } n\ge 5 \\
		\textup{Druge dvije sume: } \sum_{k=4}^{n} 2\cdot \binom{n}{k} \le \sum_{k=4}^{n}\binom{n}{k}B(k) \\
		\textup{Član po član sume: }\binom{n}{k} 2 \le \binom{n}{k} B(k) \\
		2 \le B(k), \textup{za } k \ge 4
	\end{align}
\end{proof}

\begin{proof}
	Potrebno je dokazati $B(n) \le n!$. Dokaz se provodi iz svojstava za Bellov broj (Berend, D.; Tassa, T. (2010)): 
	\[ B(n) \le \left [ \frac{0.792 n}{ln(1+n)} \right ]^n \] 
	i svojstvo faktorijela koja slijedi iz analize Stirlingove aproksimacije: \[ n!\ge \left ( \frac{n}{e} \right )^n \]
	\begin{align}
		B(n) \le \left [ \frac{0.792 n}{ln(1+n)} \right ]^n \le \left ( \frac{n}{e} \right )^n \le n! \\
		\textup{Dovoljno je pokazati: } \left [ \frac{0.792 n}{ln(1+n)} \right ]^n \le \left ( \frac{n}{e} \right )^n \\
		\left [ \frac{0.792 n}{ln(1+n)} \right ] \le \left ( \frac{n}{e} \right ) \\
		0.792 n\cdot e \le n \cdot ln(n+1) \\
		0.792\cdot e \le ln(n+1) \\
		e^{0.792\cdot e} \le n+1 \\
		8.61 -1 \le n \\
		7.61 \le n
	\end{align}
\end{proof}

Zaključak predhodne alanize je da za relativno veliki graf postoji velik broj stanja koje je nemoguće pretražiti pomoću \emph{brute force} algoritama. Analiza prostora stanja za zadani problem malo je drugačija. Kako svaki čvor ima konačan broj boja koje može poprimiti ukupan broj stanja je:

\begin{equation}
N = \prod_{i=1}^{V(G)} K_i
\end{equation}
, gdje je oznaka $K_i$ ukupan broj boja koje može poprimiti čvor s indeksom $i$. Kako je moguće dodijeliti maksimalno samo $512$ boja, možemo ograničiti $K_i$:
\begin{equation}
K_i \le M = 512
\end{equation}
\begin{equation}
N = \prod_{i=1}^{V(G)} K_i \le \prod_{i=1}^{V(G)} M = M^{V(G)}
\end{equation}

Broj stanja eksponencijalno ovisi o broju čvorova, što je isto kao i za klasičan problem bojanja grafa jako velik broj stanja. I u ovom slučaju važno je primjetiti da su neka stanja ekvivalentna. Broj jedinstvenih stanja je broj particija skupa čvorova u skupove boja, ali uz ograničenje da barem jedan čvor ima neku od zadanih $M$ boja, tada vrijedi ova nejednakost za broj različitih stanja:

\begin{equation}
N \le \sum_{i=1}^{M} S(V(G), i)
\end{equation}

Broj stanja je manji jer se poštuje ograničenje da čvorovi mogu poprimiti neku od boja u podskupu od ukupnog konstantnog broja boja $M$. Kada je $k\ge M$ tada broj stanja optimizacijskog problema je manji od broja stanja u problemu gdje se traži odgvor na pitanje da li je graf \textbf{k-obojiv}. Ovo razmatranje biti će zanimljivo i u sljedećem poglavlju gdje se problem svrstava u klase računske teorije složenosti.

\subsection{Pozadina problema iz perspektive računske teorije složenosti}

Problem odluke da li je graf \textbf{k-obojiv} spada u klasu \textbf{NP-potpunih} problema. Traženje kromatskog broja grafa $\chi(G)$ spada u klasu \textbf{NP-teških} problema. Dokazi se mogu pronaći u \cite{Garey:1974:SNP:800119.803884}. Potrebno je pokazati da postavljen optimizacijski problem bojanja grafa spada u klasu \textbf{NP-teških} problema. Problem sigurno ne može spadati u druge spomenute (P, NP, NP-potpun) klase zato jer to nije problem odluke.
Postupak dokazivanja da problem spada u klasu \textbf{NP-teških} problema ima nekoliko koraka. Prvi korak je pronalaženje nekog \textbf{NP-potpunog} ili \textbf{NP-teškog} problema koji će se koristiti u dokazu. Zatim se pokaže da je taj \emph{stari} problem moguće polinomno reducirati na problem za koji se dokazuje da je \textbf{NP-težak}. 

\begin{proof}
Odaberemo \textbf{NP-potpun problem} koji ispituje da li je graf $G$ \textbf{k-obojiv}. Da bismo rješili taj problem koristimo zadani optimizacijski problem. Ulaz u optimizacijski algoritam međuostalim su graf $G$ i ograničenja boja za svaki od čvorova. Svakom čvoru daje se skup $\{1,2,...,k\}$ s bojama. Svim bridovima daje se težina $1$. Sada se pokreće crna kutija s optimizacijskim problemom koja će rješiti taj problem. Ako je rezultat optimizacije tj. funkcije cilja $f=0$, tada je graf \textbf{k-obojiv}, inače nije. U konstantnom vremenu možemo reducirati problem, iz čega sljedi da je optimizacijski problem \textbf{NP-težak}.  
\end{proof}

Zanimljivo je primjetiti da \textbf{NP-potpun} problem k-obojivosti može imati više stanja od \textbf{NP-teškog} problema optimizacije. Što znači da je moguće da vrijeme izvršavanja optimizacije bue manje od vremena provjere k-obojivosti. Kako nije poznat odnos \emph{NP} i \emph{P} klase, za ovaj optimizacijski problem nameće se korištenje heuristika. U ovom tehničkom problemu nije potrebno do kraja minimizirati funkciju cilja, nego dobiti dovoljno malu vrijednost $f$ što će uzrokovati bolju kvalitetu mobilne mreže. 

\subsection{Posljedica "\emph{No free lunch}" teorema}

U originalnom članku, gdje se dokazuje "\emph{No free lunch}" teorem [literatura], pretpostavlja se da sve funkcije cilja imaju iste domene i kodomene. Za problem bojanja grafova to nije tako. Funkcija cilja definirana je kao: $f:V(G) \rightarrow \mathbb{R}$, ali kako računalo ima konačnu memoriju tada je funkcija zapravo definirana kao: $f:V(G) \rightarrow \mathbb{Y}$, gdje je $\mathbb{Y}$ konačni podskup realnih brojeva koji se mogu prikazati na računalu. Problem za "\emph{No free lunch}" teorem je to da veličina domene funkcije cilja ovisi o broju čvorova grafa. Broj grafova je beskonačan, jer možemo uzeti proizvoljan broj čvorova, što znači da postoji i beskonačan broj mogućih funkcija cilja. Teorem radi samo ako je broj funkcija cilja ograničen (teorem \ref{thm:nofreelunch}). Kako je na nekom računalu nemoguće prikazati proizvoljno veliki graf, broj čvorova može se ograničiti s nekom granicom $M$. Sada je moguće prebrojati sve funkcije cilja $f$.

\begin{proof}
Pitanje je koliko funkcija postoji, ako je broj članova domene ograničen s $M$.
Broj funkcija koje imaju točno $k$ elemenata u domeni ima:
	\begin{equation}
		|\mathbb{F}_k| = |\mathbb{Y}|^k
	\end{equation}
Ukupan broj funkcija je suma po svim $k$ do granice $M$:
	\begin{equation}
		|\mathbb{F}| = \sum_{k=0}^{M}|\mathbb{F}_k| = \sum_{k=0}^{M} |\mathbb{Y}|^k = ... 
	\end{equation}
\end{proof} 

\chapter{Algoritmi}

%TODO uvod, greedy - znanje o f, zasto ne samo greedy - stari rad o tome kako random ne radi prica o uniformnom podjeli boja, dokaz da greedy ne radi, 

\chapter{Infrastruktura}


\chapter{Testiranje}

\chapter{Rezultati}

\chapter{Zaključak}
Zaključak.

\bibliography{literatura}
\bibliographystyle{fer}

\begin{sazetak}
Sažetak na hrvatskom jeziku.

\kljucnerijeci{Ključne riječi, odvojene zarezima.}
\end{sazetak}

% TODO: Navedite naslov na engleskom jeziku.
\engtitle{Title}
\begin{abstract}
Abstract.

\keywords{Keywords.}
\end{abstract}

\end{document}
