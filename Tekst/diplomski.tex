\documentclass[times, utf8, diplomski, numeric]{fer}
\usepackage{booktabs}

\begin{document}

% TODO: Navedite broj rada.
\thesisnumber{656}

% TODO: Navedite naslov rada.
\title{Bojanje grafova prilagodljivim metaheurističkim postupcima}

% TODO: Navedite vaše ime i prezime.
\author{Dino Šantl}

\maketitle

% Ispis stranice s napomenom o umetanju izvornika rada. Uklonite naredbu \izvornik ako želite izbaciti tu stranicu.
\izvornik

% Dodavanje zahvale ili prazne stranice. Ako ne želite dodati zahvalu, naredbu ostavite radi prazne stranice.
\zahvala{}

\tableofcontents

\chapter{Uvod}
Uvod rada. Nakon uvoda dolaze poglavlja u kojima se obrađuje tema.

\section{Sažetci korištenih znanstvenih članaka}

\chapter{Formalni opis problema}

U ovom poglavlju opisuje se problem bojanja grafova i njegova primjena u telekomunikacijskom problemu koji se rješava. Najprije se predstavlja problem u telekomunikacijama uz pretpostavke i ograničenja. Zatim se isti problem opisuje pomoću matematičkog modela tj. problema bojanja grafova. Prvo se objašnjava općeniti problem bojanja grafova, a zatim se matematički model prilagodi konkretnom problemu.

\section{Tehnički opis problema}

Za shvaćanje problema dovoljno je pretpostaviti da postoji korisnikova oprema (što je najčešće mobilni uređaj) i bazna stanica. Bazne stanice geografski su statične, za razliku od mobilnih uređaja.
Ono što se u ovom radu naziva baznom stanizom je radio pristupna mreža \engl{Radio Access Network} (\emph{RAN}). \emph{RAN} dio je telekomunikacijskog sustava i nalazi se između opreme korisnika i jezgre mreže. To je sloj u mreži koji je zaslužan za prenošenje komunikacije između mobilnog uređaja do antene i od antene do drugog mobilnog uređaja. 

Za što kvalitetnji rad mreže potrebno je optimirati parametre \emph{RAN}-a. Jedan od parametra je \emph{scrambling} k\^{o}d. \emph{Scrambling} k\^{o}d međuostalim služi kako bi mobilni uređaj mogao razlikovati bazne stanice. Zbog toga je potrebno svakoj baznoj stanici pridružiti različit k\^{o}d. Problem je u tome što je dostupno samo 512 različitih kodova. To znači da neke bazne stanice moraju imati isti k\^{o}d (ako se u sustavu nalazi više od 512 baznih stanica). Ako neke dvije bazne stanice imaju isti k\^{o}d i mobilni uređaj se nalazi u dosegu signala od obje stanice, tada ne zna s kojom baznom stanicom komunicira te pada kvaliteta usluge.

Potrebno je dodjeliti \emph{scrambling} k\^{o}dove tako da dvije bazne stanice koje mogu biti istovremeno vezane za jedan mobilni uređaj nemaju isti k\^{o}d ili ako je to nemoguće smanjiti, dodjeliti k\^{o}dove tako da je negativan utjecaj na kvalitetu usluge što manji. Postoje još neka tehnička ograničenja koja je potrebno uvažiti, a biti će navedena u nastavku.



\section{Matematički opis problema}

\chapter{Infrastruktura}

\chapter{Algoritmi}

\chapter{Testiranje}

\chapter{Rezultati}

\chapter{Zaključak}
Zaključak.

\bibliography{literatura}
\bibliographystyle{fer}

\begin{sazetak}
Sažetak na hrvatskom jeziku.

\kljucnerijeci{Ključne riječi, odvojene zarezima.}
\end{sazetak}

% TODO: Navedite naslov na engleskom jeziku.
\engtitle{Title}
\begin{abstract}
Abstract.

\keywords{Keywords.}
\end{abstract}

\end{document}
