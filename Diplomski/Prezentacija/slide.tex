\documentclass[utf8]{beamer}
\usetheme{Dresden}
\usepackage[T1]{fontenc}
\usepackage{tikz}
\usepackage{caption}
\usepackage{subfigure}


%dodatak za programski kod
\usepackage{listings}
\usepackage{color}
\usepackage{setspace}
\definecolor{dkgreen}{rgb}{0,0.6,0}
\definecolor{gray}{rgb}{0.5,0.5,0.5}
\definecolor{mauve}{rgb}{0.58,0,0.82}

\lstset{frame=tb,
  language=Java,
  aboveskip=3mm,
  belowskip=3mm,
  showstringspaces=false,
  columns=flexible,
  basicstyle={\small\ttfamily},
  numbers=left,
  numberstyle=\small\color{gray},
  keywordstyle=\color{blue},
  commentstyle=\color{dkgreen},
  stringstyle=\color{mauve},
  breaklines=true,
  breakatwhitespace=true,
  tabsize=2
}


% Postavljanje fonta
\if@fonttimes\RequirePackage{times} \fi
\if@fontlmodern\RequirePackage{lmodern} \fi

\usecolortheme{dolphin}

%prikaz broja slajda
\expandafter\def\expandafter\insertshorttitle\expandafter{%
	\insertshorttitle\hfill%
	\insertframenumber\,/\,\inserttotalframenumber}

\newcommand{\engl}[1]{(engl.~\emph{#1})}
\newcommand{\putat}[3]{\begin{picture}(0,0)(0,0)\put(#1,#2){#3}\end{picture}}

\title[Simbolička regresija]{Simbolička regresija}

\author[Dino Šantl]{Seminar \newline\newline Dino Šantl\newline Mentor: Prof. dr. sc. Domagoj Jakobović}

\institute{Fakultet elektrotehnike i računarstva}
\date{Zagreb, lipanj 2013.}
\begin{document}

\begin{frame}
\titlepage
\end{frame}

\section{Uvod}
\begin{frame}{Uvod}
\begin{itemize}
\item Postupak pronalaženja matematičkog izraza iz empirijskih podataka
\begin{figure}
\centering
\includegraphics[width=7cm]{slike/sinus.png}
\end{figure}
\end{itemize}
\end{frame}


\begin{frame}{Uvod}
	\begin{itemize}
		\item Osim traženja parametra modela traži se i sam model 
		\item Model nije pretpostavljen kao npr. kod linearne regresije
	\end{itemize}
	\vspace{20px}
	Linearna regresija:
	\begin{equation} 
	h(x) = \mathbf{w_{1}}\cdot x^{2} + \mathbf{w_{2}}\cdot x + \mathbf{w_{0}}
	\end{equation}
	Simbolička regresija:
	\begin{equation} 
	h(x) = \mathbf{2\cdot ln(x) + 0.5\cdot sin(x)} 
	\end{equation}
	\center
	ili
	\begin{equation} 
	h(x) = \mathbf{e^{x} \cdot cos^{2}(x)} 
	\end{equation}
\end{frame}

\section{Evolucijski algoritam}
\begin{frame}{Evolucijski algoritam}
	\begin{itemize}
		\item Prostor stanja prevelik za iscrpno pretraživanje \engl{brute force} - heuristika $\rightarrow$ evolucijski algoritam
		\item Evolucijski algoritam
		\begin{itemize}
			\item \textbf{jedinka}
			\item populacija
			\item selekcija
			\item križanje 
			\item mutacija
			\item dobrota jedinke - \textbf{račun pogreške}
		\end{itemize}
	\end{itemize}
\end{frame}

\begin{frame}{Evolucijski algoritam}
	\begin{itemize}
	\item Genetsko programiranje
	\pause
	\item Gramatička evolucija
	\pause
	\item Analitičko programiranje
	\end{itemize}
\end{frame}


\begin{frame}{Genetsko programiranje}
	\begin{itemize}
		\item Jedinka prikazana stablom: $\sqrt{y} \cdot x+sin(3.14\cdot x)$
	\end{itemize}
	\begin{figure}
		\centering
		\includegraphics[width=5cm]{slike/sintaksno_stablo.pdf}
	\end{figure}
\end{frame}

\begin{frame}{Gramatička evolucija}
	\begin{itemize}
		\item Jedinka prikazana u binarnom obliku - 0011 1110 0101 1000
		\item Formira se niz iz kojeg se gradi izraz - $[3, 14, 5, 8]$
		\item Izraz se gradi na temelju zadane gramatike (npr. BNF oblik)
	\end{itemize}
\pause
\begin{tabular}{r l l c}                       
	<izraz> & ::= & <izraz> <operator> <izaraz> & (0)\\
	  & | & <var> & (1)\\
	  & | & <func>(<izraz>) & (2)\\
	 <operator> & ::= & + & (0)\\
	  & | & - & (1)\\
	<var> & ::= & X & (0)\\
	& | & 3.14 & (1)\\
	<func> & ::= & Sin & (0)\\
	& | & Log & (1)\\
\end{tabular}

\end{frame}

\begin{frame}{Analitičko programiranje}
	\begin{itemize}
		\item Jedinka zapisana pomoću niza - $[3, 14, 42, 5]$
		\item Definira se skup u kojem se nalaze operatori, funkcije, varijable i konstante
		\item Na temelju broja u nizu određuje se element u matematičkom izrazu
		\item \{ +, -, *, / , sin, cos, sqrt, X, Y, 3.14\}
 
	\end{itemize}
\end{frame}

\section{Problemi}
\begin{frame}{Problemi u simboličkoj regresiji}
\begin{itemize}
\item Predugi izrazi - regularizacija
\item Nedefiniran izlaz funkcije - \textbf{intervalna aritmetika}
\item Konstante - \textbf{linearno skaliranje}
\end{itemize}
\end{frame}

\begin{frame}{Intervalna aritmetika}
	\begin{itemize}
		\item Raniji pokušaji: sigurni operatori: $f(x)=\frac{1}{x-5}$, $x=5$?
		\pause
		\item Prikaz jedinke pomoću stabla
		\item Računa se izlaz generirane funkcije ali u intervalnoj aritmetici
		\item Ako je interval beskonačan ili se pojavi kompleksni broj potrebno je obrisati podstablo
		\item Korak prije računanja pogreške
	\end{itemize}
\end{frame}

\begin{frame}{Intervalna aritmetika}
	\begin{figure}
		\centering
		\includegraphics[width=7cm]{slike/interval.pdf}
	\end{figure}
\end{frame}

\begin{frame}{Linearno skaliranje}
	\begin{itemize}
		\item Funkcija ima "dobar" oblik ali nije dobro skalirana
		\item Primjer: $h(x) = 12.42\cdot x^{2} + 10^{8}$
		\pause
		\item Izraz se zapiše u obliku $h^{'}(x)=\mathbf{a}\cdot h(x) + \mathbf{b}$		
		\item Traže se konstante $\mathbf{a}$ i $\mathbf{b}$ koje minimiziraju srednju kvadratnu pogrešku (dobrota jedinke)
	\end{itemize}
\begin{equation}
E = \frac{1}{N}\sum_{i=1}^{N}(y_{i}-(a\cdot h(\mathbf{x_{i}})+b))^2
\end{equation}
\end{frame}


\section{Korištenje u praksi}

\begin{frame}{Izlučivanje fizikalnih zakona iz empirijskih podataka}
\begin{figure}
\centering
\includegraphics[height=1.5cm]{slike/oscilator.pdf}
\end{figure}
\begin{figure}
\centering
\includegraphics[height=2.5cm]{slike/sinus.png}
\end{figure}

Bez znanja fizike:
\begin{equation}
a-0.05v-7.8x=0
\end{equation}

\end{frame}


\begin{frame}{Izlučivanje fizikalnih zakona iz empirijskih podataka}
\begin{itemize}
\item Empirijski podaci
\pause
\item Numeričko računanje parcijalnih derivacija $\frac{\Delta y}{\Delta x}$
\pause
\item Simboličkom regresijom traži se funkcija $f(x,y)$ koja najbolje opisuje parcijalne derivacije: $\frac{\delta y}{\delta x}=\frac{\frac{\partial f}{\partial x}}{\frac{\partial f}{\partial y}}$
\pause
\item Račun pogreške:
\begin{equation}
E = -\frac{1}{N} \sum_{i=1}^{N} log\left [1+abs(\frac{\delta y_{i}}{\delta x_{i}}-\frac{\Delta y_{i}}{\Delta x_{i}}) \right ]
\end{equation}
\end{itemize}
\end{frame}

\begin{frame}{Izlučivanje fizikalnih zakona iz empirijskih podataka}
\begin{itemize}
\item Zanemarivanje nekih funkcija $\rightarrow$ aproksimacije
\pause
\item Računanje traje od par sati do nekoliko dana
\pause
\item Pokazuje se da je točno rješenje u trenutku kada se dogodi veliki skok u pogrešci
\end{itemize}
\end{frame}


\section{Pitanja}
\begin{frame}{}
\center
\Huge{Hvala na pažnji!}
\end{frame}

\begin{frame}{}
\center
\Huge{Pitanja?}
\begin{figure}
\centering
\includegraphics[width=5cm]{slike/pitanja.png}
\end{figure}
\end{frame}



\end{document}