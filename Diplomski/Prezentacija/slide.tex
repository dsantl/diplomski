\documentclass[utf8]{beamer}
\usetheme{Dresden}
\usepackage[T1]{fontenc}
\usepackage{tikz}
\usepackage{caption}
\usepackage{subfigure}


%dodatak za programski kod
\usepackage{listings}
\usepackage{color}
\usepackage{setspace}
\definecolor{dkgreen}{rgb}{0,0.6,0}
\definecolor{gray}{rgb}{0.5,0.5,0.5}
\definecolor{mauve}{rgb}{0.58,0,0.82}

\lstset{frame=tb,
  language=Java,
  aboveskip=3mm,
  belowskip=3mm,
  showstringspaces=false,
  columns=flexible,
  basicstyle={\small\ttfamily},
  numbers=left,
  numberstyle=\small\color{gray},
  keywordstyle=\color{blue},
  commentstyle=\color{dkgreen},
  stringstyle=\color{mauve},
  breaklines=true,
  breakatwhitespace=true,
  tabsize=2
}

\renewcommand{\figurename}{Slika}

% Postavljanje fonta
\if@fonttimes\RequirePackage{times} \fi
\if@fontlmodern\RequirePackage{lmodern} \fi

\usecolortheme{sidebartab}

%prikaz broja slajda
\expandafter\def\expandafter\insertshorttitle\expandafter{%
	\insertshorttitle\hfill%
	\insertframenumber\,/\,\inserttotalframenumber}

\newcommand{\engl}[1]{(engl.~\emph{#1})}
\newcommand{\putat}[3]{\begin{picture}(0,0)(0,0)\put(#1,#2){#3}\end{picture}}

\renewcommand\tablename{Tablica}

\title[Bojanje grafova prilagodljivim metaheurističkim postupcima]{Bojanje grafova prilagodljivim metaheurističkim postupcima}

\author[Dino Šantl]{Diplomski rad \newline\newline Dino Šantl\newline Mentor: Prof. dr. sc. Domagoj Jakobović}

\institute{Fakultet elektrotehnike i računarstva}
\date{Zagreb, srpanj 2014.}
\begin{document}

\begin{frame}
\titlepage
\end{frame}

\section*{}
\begin{frame}{Sadržaj}
	\begin{enumerate}
		\item Uvod - definicija i analiza problema
		\item Algoritmi i metode
		\item Programsko rješenje
		\item Rezultati
		\item Zaključak
	\end{enumerate}
\end{frame}


\section{Uvod}
\begin{frame}{Uvod}
	\begin{itemize}
		\item Korištenje metaheuristika u stvarnom problemu
		\pause
		\item Cilj: Povećanje kvalitete mobilne mreže
		\pause
		\item \emph{Scrambling Code} -- prepoznavanje baznih stanica 
		\pause
		\item Problem: $\sim 100000$ baznih stanica i \emph{samo} $512$ k\^{o}dova
		\pause
		\item Pokušaj rješenja: uniformna raspodjela kodova - \textbf{neuspješno} 
	\end{itemize}
\end{frame}

\begin{frame}{Modeliranje problema}
	\begin{itemize}
		\item Bazne stanice modeliraju se čvorovima
		\pause
		\item Udaljenosti između baznih stanica -- težinski bridovi
		\pause
		\item Model -- težinski graf
		\pause
		\item Postavljanje k\^{o}dova - \textbf{problem bojanja težinskih grafova}
	\end{itemize}
	\begin{figure}[h]
  \caption{Primjer težinskog grafa}
  \centering
    \includegraphics[width=0.4\textwidth]{../Tekst/Slike/simple_example.png}
	\end{figure}
\end{frame}

\begin{frame}{Definicija problema}
	\begin{itemize}
		\item Ulaz: početno stanje mreže (informacije o čvorovima)
		\pause
		\item Obojati graf s ciljem minimizacije sume težina konfliktnih bridova
		\pause
		\item Najmanje $33\%$ čvorova mora imati svoju početnu boju
	\end{itemize}
	\pause
	\begin{figure}[h]
  \caption{Primjer bojanja težinskog grafa - greška: $10$}
  \centering
    \includegraphics[width=0.4\textwidth]{../Tekst/Slike/simple_example_color.png}
	\end{figure}
\end{frame}

\begin{frame}{Analiza problema}
	\begin{itemize}
		\item Za broj stanja (različitih bojanja): $N \le \sum_{i=1}^{M} S(V(G), i)$
		\pause
		\item Za $M=V(G)$ vrijedi: $N=B(M)$
		\begin{itemize}
			\item $2^M \le B(M) \le M!$
		\end{itemize}
		\pause
		\item Broj stanja prevelik za iscrpnu pretragu!
	\end{itemize}
\end{frame}

\begin{frame}{Analiza problema}
	\begin{itemize}
		\item Optimizacijski problem (nije problem odluke)
		\pause
		\item Problem je \textbf{NP-težak}
		\pause
		\item \emph{No free lunch} teorem?
		\pause
		\item Nameće se korištenje (meta)heuristika
	\end{itemize}
\end{frame}

\section{Algoritmi}
\begin{frame}{Korišteni algoritmi}
	\begin{itemize}
		\item Pohlepni algoritam
		\pause
		\item Agentski algoritam
		\pause
		\item Simulirano kaljenje
		\pause
		\item Evolucijska strategija
		\pause
		\item Genetsko kaljenje
	\end{itemize}
\end{frame}

\begin{frame}{Pohlepni algoritam}
	\begin{itemize}
		\item Obilazak čvorova i dodjela boja
		\pause
		\item Pronalazak relativno dobre vrijednosti funkcije cilja
		\pause
		\item Vrste sortiranja: COL, FIT, LDO, SDO, SDOLDO, RND
		\pause
		\item Vrste bojanja: ABW, MC, MF, SWAP, RND
		\pause
		\item Pohlepan algoritam ne pronađe nužno globalni minimum
	\end{itemize}
\end{frame}

\begin{frame}{Agentski algoritam}
	
	\putat{90}{30}{ \includegraphics[width=0.07\textwidth]{../Tekst/Slike/agent.png}}
	
	\begin{itemize}
		\item Agent - nalazi se na nekom čvoru i donosi odluku o novoj boji
		\pause
		\item Boduje se prema najvećoj težini konfliktnog brida
		\pause
		\item Sortiranje agenata prema bodovima
		\pause
		\item Odluka za agenta: 
			\begin{itemize}
				\item Ostajanje na čvoru
				\item Slučajni susjed
				\item \emph{Najteži} susjed
			\end{itemize}
		\end{itemize}
\end{frame}

\begin{frame}{Simulirano kaljenje}
	\begin{itemize}
		\item Dva rješenja: staro $X$ i novo $S$
		\pause
		\item Ako je novo bolje sigurno se prihvaća
		\pause
		\item Ako je rješenje lošije prihvaća se s određenom vjerojatnošću
		\begin{itemize}
			\item $P(X=S) = e^{-\frac{f(S)-f(X)}{T}}$
		\end{itemize}
		\pause
		\item Temperatura $T$ pada s vremenom
	\end{itemize}
\end{frame}

\begin{frame}{Evolucijska strategija}
	\begin{itemize}
		\item Genetski algoritam -- smisao operatora križanja? -- loši rezultati
		\pause
		\item Evolucijska strategija $(\mu, \lambda)$ -- kreiraju se nova rješenja, stara se brišu
		\pause
		\item Samo operator mutacije - vrste bojanja
		\end{itemize}
\end{frame}

\begin{frame}{Genetsko kaljenje}
	\begin{itemize}
		\item Kombinacija simuliranog kaljenja i genetskog algoritma
		\pause
		\item Komunikacija između \emph{jedinki} modelira se zajedničkom energijom koja se ravnopravno podijeli
		\pause
		\item Energija: koliko rješenje može biti lošije, a da se prihvati
	\end{itemize}
\end{frame}


\begin{frame}{Tuneliranje i postotak bojanja}
	\begin{itemize}
		\item \emph{SWAP} vrsta bojanja
		\pause
		\begin{itemize}
			\item Odabire se točka u prostoru čija vrijednost funkcije cilja približno jednaka trenutnoj
		\end{itemize}
		\pause
		\item Postotak bojanja - u operatore se ugrađuje preferiranje početne boje
		\begin{itemize}
			\item Agresivni i blagi operatori
		\end{itemize}
	\end{itemize}
\end{frame}

\begin{frame}{Minimizacija postotka promjene početne boje}
\begin{figure}[h]
  \centering
  \includegraphics[width=0.8\textwidth]{../Tekst/Slike/kondenz_iter_per_.png}
\end{figure}
\end{frame}


\begin{frame}{Izlučivanje znanja}
	\begin{itemize}
		\item Da li postoji funkcija $f(\textup{COL}, \textup{FIT}, \textup{LDO}, \textup{SDO})$ po kojoj se čvorovi sortiraju, a da je rezultat pohlepnog algoritma optimalan?
		\pause
		\item Slučajno sortirati čvorove
		\pause
		\item Obojati nekom vrstom bojanja
		\pause
		\item Uzeti najbolje rezultate kao ulaz za regresiju
		\pause
		\item Korištenje regresije (odabir modela ili simbolička regresija)
		\pause
		\item Fokus je na čvorovima (ne na grafu)
	\end{itemize}
\end{frame}

\begin{frame}{Izlučivanje znanja}
	\begin{itemize}
		\item Rezultati nisu pokazali postojanje funkcije koja kombinira vrste sortiranja
		\item Najveći utjecaj ima $SDO$, a nakon toga $LDO$ -- što se poklapa s prijašnjim istraživanjima
		\item Ako se greška minimizira lokalno $MF$, tada i $FIT$ ima utjecaj
	\end{itemize}
\end{frame}



\section{Programsko rješenje}
\begin{frame}{Programsko rješenje}
	\begin{itemize}
		\item \emph{Java} programski jezik
		\pause
		\item Programsko okruženje za razvoj algoritama za ovaj specifičan problem
		\pause
		\item Razni formati za čitanje grafova i spremanje rezultata, podrška za populacijske algoritme...
	\end{itemize}
\end{frame}


\section{Rezultati}
\begin{frame}{Minimizacija funkcije cilja}
\begin{table}[htb]
	\centering
	\begin{tabular}{|l|r|r|} \hline
	Graf & Funkcija cilja & Postotak promjene početne boje \\ \hline \hline
	Tokai & 1195.255 & 65\% \\ \hline
	Kansai & 4104.29 & 66\% \\ \hline
	\end{tabular}
\end{table}

\end{frame}


\begin{frame}{Usporedba algoritama}
\begin{figure}[h]
  \centering
  \includegraphics[width=1.0\textwidth]{../Tekst/Slike/tokai-A.pdf}
\end{figure}
\end{frame}

\begin{frame}{Usporedba algoritama}
\begin{table}[htb]
	\centering
	\caption{Standardna devijacija}
	\begin{tabular}{|l|r|} \hline
	Algoritam & Standardna devijacija \\ \hline \hline
	Agentski algoritam  & 166.5007100085  \\ \hline
	Simulirano kaljenje & 2.3047581941 \\ \hline
	Evolucijska strategija & 7.855956339 \\ \hline
	Genetsko kaljenje & 0.1745653573 \\ \hline
	\end{tabular}
\end{table}
\end{frame}


\section{Zaključak}
\begin{frame}{Zaključak}
	\begin{itemize}
		\item Problem se može efikasno rješavati metaheuristikama
		\pause
		\item Važno je ugraditi specifične operatore i namjestiti parametre
		\pause
		\item Rezultati pokazuju da se ovakav pristup može korisititi za klasičan problem bojanja grafova
		\pause
		\item Moguća poboljšanja i ubrzanja
	\end{itemize}
\end{frame}


\section*{}
\begin{frame}{}
\center
\Huge{Hvala na pažnji!}
\end{frame}

\begin{frame}{}
\center
\Huge{Pitanja?}
\begin{figure}[h]
  \centering
    \includegraphics[width=0.4\textwidth]{../Tekst/Slike/pitanja.png}
	\end{figure}
\end{frame}


\end{document}