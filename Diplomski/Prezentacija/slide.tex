\documentclass[utf8]{beamer}
\usetheme{Dresden}
\usepackage[T1]{fontenc}
\usepackage{tikz}
\usepackage{caption}
\usepackage{subfigure}


%dodatak za programski kod
\usepackage{listings}
\usepackage{color}
\usepackage{setspace}
\definecolor{dkgreen}{rgb}{0,0.6,0}
\definecolor{gray}{rgb}{0.5,0.5,0.5}
\definecolor{mauve}{rgb}{0.58,0,0.82}

\lstset{frame=tb,
  language=Java,
  aboveskip=3mm,
  belowskip=3mm,
  showstringspaces=false,
  columns=flexible,
  basicstyle={\small\ttfamily},
  numbers=left,
  numberstyle=\small\color{gray},
  keywordstyle=\color{blue},
  commentstyle=\color{dkgreen},
  stringstyle=\color{mauve},
  breaklines=true,
  breakatwhitespace=true,
  tabsize=2
}

\renewcommand{\figurename}{Slika}

% Postavljanje fonta
\if@fonttimes\RequirePackage{times} \fi
\if@fontlmodern\RequirePackage{lmodern} \fi

\usecolortheme{dolphin}

%prikaz broja slajda
\expandafter\def\expandafter\insertshorttitle\expandafter{%
	\insertshorttitle\hfill%
	\insertframenumber\,/\,\inserttotalframenumber}

\newcommand{\engl}[1]{(engl.~\emph{#1})}
\newcommand{\putat}[3]{\begin{picture}(0,0)(0,0)\put(#1,#2){#3}\end{picture}}

\title[Bojanje grafova prilagodljivim metaheurističkim postupcima]{Bojanje grafova prilagodljivim metaheurističkim postupcima}

\author[Dino Šantl]{Diplomski rad \newline\newline Dino Šantl\newline Mentor: Prof. dr. sc. Domagoj Jakobović}

\institute{Fakultet elektrotehnike i računarstva}
\date{Zagreb, srpanj 2014.}
\begin{document}

\begin{frame}
\titlepage
\end{frame}

\section*{}
\begin{frame}{Sadržaj}
	\begin{enumerate}
		\item Uvod - definicija i analiza problema
		\item Algoritmi i metode
		\item Programsko rješenje
		\item Rezultati
		\item Zaključak
	\end{enumerate}
\end{frame}


\section{Uvod}
\begin{frame}{Uvod}
	\begin{itemize}
		\item Kako mobilni uređaj ralikuje bazne stanice?
		\pause
		\item \emph{Scrambling Code}
		\pause
		\item Problem: $\sim 100000$ baznih stanica i \emph{samo} $512$ k\^{o}dova
		\pause
		\item Pokušaj rješenja: uniformna raspodijela kodova - \textbf{neuspješno} 
	\end{itemize}
\end{frame}

\begin{frame}{Modeliranje problema}
	\begin{itemize}
		\item Bazne stanice modeliraju se čvorovima
		\item Udaljenosti između baznih staniza -- težinski bridovi
		\item Model -- težinski graf
		\item Postavljanje k\^{o}dova - \textbf{problem bojanja težinskih grafova}
	\end{itemize}
	\begin{figure}[h]
  \caption{Primjer težinskog grafa}
  \centering
    \includegraphics[width=0.4\textwidth]{../Tekst/Slike/simple_example.png}
	\end{figure}
\end{frame}

\begin{frame}{Definicija problema}
	\begin{itemize}
		\item Ulaz:
			\begin{itemize}
				\item Popis i definicija domena (skupova) boja
				\pause
				\item Popis čvorova
					\begin{itemize}
						\item Oznaka čvora
						\item Vrsta čvora - (A, B, C)
						\item Oznaka domene boja
						\item Početna boja čvora
						\item Oznaka promjenjivosti čvora
					\end{itemize}
				\pause
				\item Popis bridova
			\end{itemize}
		\pause
		\item Izlaz:
			\begin{itemize}
				\item Oznaka čvora i boja čvora
			\end{itemize}
	\end{itemize}
\end{frame}

\begin{frame}{Definicija problema}
	\begin{itemize}
		\item Obojati graf s ciljem minimizacije sume težina konfliktnih bridova
		\item Potrebno je poštivati pravilo: manje od $66\%$ čvorova mora imati svoju početnu boju
	\end{itemize}
	\begin{figure}[h]
  \caption{Primjer bojanja težinskog grafa - greška: $10$}
  \centering
    \includegraphics[width=0.4\textwidth]{../Tekst/Slike/simple_example_color.png}
	\end{figure}
\end{frame}

\begin{frame}{Analiza problema}
	\begin{itemize}
		\item Za broj stanja: $N \le \sum_{i=1}^{M} S(V(G), i)$
		\pause
		\item Za $M=V(G)$ vrijedi: $N=B(M)$
		\begin{itemize}
			\item $2^M \le B(M) \le M!$
		\end{itemize}
		\pause
		\item Broj stanja prevelik za iscrpnu pretragu!
	\end{itemize}
\end{frame}

\begin{frame}{Analiza problema}
	\begin{itemize}
		\item Optimizacijski problem (nije problem odluke)
		\pause
		\item Traži se minimum funkcije cilja: sume konfliktnih bridova
		\pause
		\item Problem je \textbf{NP-težak}
		\pause
		\item \emph{No free lunch} teorem?
		\pause
		\item Nameće se korištenje (meta)heuristika
	\end{itemize}
\end{frame}

\section{Algoritmi}
\begin{frame}{Korišteni algoritmi}
	\begin{itemize}
		\item Pohlepni algoritam
		\pause
		\item Agentski algoritam
		\pause
		\item Simulirano kaljenje
		\pause
		\item Evolucijska strategija
		\pause
		\item Genetsko kaljenje
	\end{itemize}
\end{frame}

\begin{frame}{Pohlepni algoritam}
	\begin{itemize}
		\item Obilazak čvorova i dodijela boja
		\pause
		\item Vrste sortiranja: COL, FIT, LDO, SDO, SDOLDO, RND
		\pause
		\item Vrste bojanja: ABW, MC, MF, RND, SWAP
		\pause
		\item Pohlepan algoritam nije ispravan!
	\end{itemize}
\end{frame}

\begin{frame}{Agentski algoritam}
	\begin{itemize}
		\item Agent - nalazi se na nekom čvoru
		\item Boduje se prema najvećem svojem konfliktnom bridu
		\item Sortiranje agenata prema bodovima
		\item Odluka za agenta: 
			\begin{itemize}
				\item Ostajanje na čvoru
				\item Slučajni susjed
				\item \emph{Najteži} susjed
			\end{itemize}
		\end{itemize}
\end{frame}

\begin{frame}{Simulirano kaljenje}
	\begin{itemize}
		\item Dva rješenja: staro i novo
		\pause
		\item Ako je novo bolje sigurno se prihvaća
		\pause
		\item Ako je rješenje lošje prihvaća se s određenom vjerojatnošću
	\end{itemize}
\end{frame}

\begin{frame}{Evolucijska strategija}
	\begin{itemize}
		\item Genetski algoritam -- smisao operatora križanja? -- loši rezultati
		\pause
		\item Evolucijska strategija $(\mu, \lambda)$ -- kreiraju se nova rješenja, stara se brišu
		\pause
		\item Samo operator mutacije
		\end{itemize}
\end{frame}

\begin{frame}{Genetsko kaljenje}
	\begin{itemize}
		\item Kombinacija simuliranog kaljenja i genetskog algoritma
		\pause
		\item Komunikacija između \emph{jedinki} modelira se zajedničkom energijom
		\pause
		\item Energija: koliko rješenje može biti lošije a da se prihvati
	\end{itemize}
\end{frame}


\begin{frame}{Tuneliranje i postotak bojanja}
	\begin{itemize}
		\item \emph{SWAP} vrsta bojanja
		\pause
		\item Odabire se točka u prostoru čija vrijednost funkcije cilja približno jednaka trenutnoj
		\pause
		\item Funkcija cilja u teoriji može samo padati
		\pause
		\item Postotak bojanja - u operatore se ugrađuje preferiranje početne boje
	\end{itemize}
\end{frame}

\begin{frame}{Izlučivanje znanja}
	\begin{itemize}
		\item Da li postoji funkcija $f(\textup{COL}, \textup{FIT}, \textup{LDO}, \textup{SDO})$ po kojoj se čvorovi sortiraju, a da je rezultat pohlepnog algoritma optimalan?
		\pause
		\item Korištenje regresije (odabir modela ili simbolička regresija)
		\pause
		\item Slučajno sortirati čvorove
		\pause
		\item Obojati nekom vrstom bojanja
		\pause
		\item Uzeti najbolje rezultate kao ulaz za regresiju
		\pause
		\item Fokus je na čvorovima (ne na grafu)
	\end{itemize}
\end{frame}


\section{Programsko rješenje}
\begin{frame}{Programsko rješenje}
	\begin{itemize}
		\item \emph{Java} programski jezik
		\pause
		\item Programsko okruženje za razvoj algoritama za ovaj specifičan problem
		\pause
		\item Razni formati za čitanje grafova i spremanje rezultata, podrška za populacijske algoritme...
	\end{itemize}
\end{frame}


\section{Rezultati}
\begin{frame}{Minimizacija funkcije cilja}
\end{frame}

\begin{frame}{Minimizacija postotka promjene početne boje}
\end{frame}

\begin{frame}{Traženje optimalne funkcije sortiranja}
\end{frame}

\begin{frame}{Rezultati DIMACS baze}
\end{frame}

\section{Zaključak}
\begin{frame}{Zaključak}
\end{frame}


\section*{}
\begin{frame}{}
\center
\Huge{Hvala na pažnji!}
\end{frame}

\begin{frame}{}
\center
\Huge{Pitanja?}

\end{frame}


\end{document}